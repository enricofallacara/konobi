\documentclass{beamer}

\usepackage{blindtext}

\usetheme{Patorikku}

\title{Konobi game}
\subtitle{Software Development Method Project}
\author{Fallacara E., Indri P., Pigozzi F.}
\date{}


\begin{document}
  % Per decidere se mostrare il numero di slide oppure no.
	\setcounter{showSlideNumbers}{0}

  % Pagina di titolo.
	\frame{\titlepage}

  % Azzera il contatore dei frame e inizia a mostrarlo.
	\setcounter{framenumber}{0}
	\setcounter{showSlideNumbers}{1}



  \begin{frame}{Goal}

    %\begin{tcolorbox}
    \begin{center}
      The goal of our project is to implement the \textbf{Konobi game} in Java, giving also the user the opportunity to choose between two interfaces: \textbf{console version} or \textbf{GUI version}
    \end{center}
    %%\end{tcolorbox}

  \end{frame}



  \begin{frame}{Konobi}

    Konobi is a drawless connection game for two players: \textbf{Black}
    and \textbf{White}. It's played on the a square board, which is initially empty. 

    \vspace{1em}

    The top and bottom edges of the board are coloured black; the left and right edges are coloured white.

    \begin{figure}
      \includegraphics[width=3.8cm, height=3.8cm]{img/board.jpg}
    \end{figure}

  \end{frame}



  \begin{frame}{Konobi Rules}

    \textbf{Starting with Black}, the players take turns placing stones of their own color on empty points of the board, one stone per turn.

    \vspace{1em}

    Two like-coloured stones are \textbf{strongly connected} if they are orthogonally adjacent to each other, and \textbf{weakly connected} if they are diagonally adjacent to each other without sharing any strongly connected neighbour.

    \vspace{1em}

    It's \textbf{illegal} to make a weak connection to a certain stone unless it's impossible to make a placement which is both strongly connected to that stone and not weakly connected to another.

  \end{frame}



  \begin{frame}{Legal and Illegal Moves}

    % Mettere margine più grande per la color box.

    \begin{figure}
      \centering
      \fcolorbox{black}{white}{\includegraphics[width=0.55\textwidth]{img/moves.jpg}}
    \end{figure}

  \end{frame}



  \begin{frame}{Konobi rules Cont.}

    It's also \textbf{illegal} to form a \textbf{crosscut}, i.e., a 2x2 pattern of stones consisting of two weakly connected Black stones and two weakly connected White stones.

    \vspace{1em}

    \begin{figure}
      \fcolorbox{black}{white}{\includegraphics[width=0.25\textwidth]{img/crosscut.jpg}}
    \end{figure}

    \vspace{1em}

    If a player can't make a move on his turn, he must \textbf{pass}. Passing is otherwise not allowed. There will always be a move available to at least one of the players.

  \end{frame}



  \begin{frame}{Konobi rules Cont.}

    The \textbf{pie rule} is used in order to make the game fair. This means that White will have the option, on his first turn only, to change sides instead of making a regular move.

    \vspace{3em}

    The game is \textbf{won} by the player who completes a chain of his color touching the two opposite board edges of his color. \textbf{Draws are not possible}.

  \end{frame}



\end{document}
